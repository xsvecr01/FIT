\documentclass[czech,a4paper,11pt,twocolumn]{article}
\usepackage[czech]{babel}
\usepackage[left=1.5cm,text={18cm, 25cm},top=2.5cm]{geometry}
\usepackage[IL2]{fontenc}
\usepackage{times}

\usepackage{amsmath}
\usepackage{amsthm}
\usepackage{amsfonts}
\usepackage{stackrel}

\theoremstyle{definition}
\newtheorem{definice}{Definice}
\theoremstyle{definition}
\newtheorem{veta}{Věta}
\theoremstyle{definition}
\newtheorem{dukaz}{Důkaz}

\begin{document}
\begin{titlepage}
\begin{center}
\Huge
\textsc{Fakulta informačních technologií}
\\
\textsc{Vysoké učení technické v~Brně\\}
\vfill
\LARGE
Typografie a publikování - 2. projekt
\\
\LARGE
Sazba dokumentů a matematických výrazů\\
\vfill
\Large{2020}\hfill Radek Švec
\end{center}
\end{titlepage}

\section*{Úvod} 
V~této úloze si vyzkoušíme sazbu titulní strany, matematických vzorců, prostředí a dalších textových struktur obvyklých pro technicky zaměřené texty (například rovnice (\ref{rovnice2}) nebo Definice \ref{definice2} na straně \pageref{definice2}). Pro vytvoření těchto odkazů používáme příkazy \verb|\label|, \verb|\ref| a \verb|\pageref|.

Na titulní straně je využito sázení nadpisu podle optického středu s~využitím zlatého řezu. Tento postup byl probírán na přednášce. Dále je použito odřádkování se zadanou relativní velikostí 0.4em a 0.3em.

\section{Matematický text}
Nejprve se podíváme na sázení matematických symbolů a~výrazů v~plynulém textu včetně sazby definic a~vět s~využitím balíku \texttt{amsthm}. Rovněž použijeme poznámku pod čarou s použitím příkazu \verb|\footnote|. Někdy je vhodné použít konstrukci \verb|${}$| nebo \verb|\mbox{}| která říká, že (matematický) text nemá být zalomen. V~následující definici je nastavena mezera mezi jednotlivými položkami \verb|\item| na 0.05em.

\begin{definice}
\label{definice1} Turingův stroj \emph{(TS) je definován jako šestice tvaru $M=(Q,\Sigma,\Gamma,\delta,q_{0}, q_{F})$, kde:}
\begin{itemize}
\itemsep0.05em
\item \emph{$Q$ je konečná mžnožina} vnitřních (řídicích) stavů,
\item \emph{$\Sigma$ je konečná mžnožina symbolů nazývaná} vstupní abeceda, $\Delta \notin \Sigma$,
\item \emph{$\Gamma$ je konečná mžnožina symbolů, $\Sigma \subset \Gamma, \Delta \in \Gamma$, nazývaná} pásková abeceda,
\item \emph{$\delta : (Q \backslash \{q_{F}\}) \times \Gamma \rightarrow  Q \times (\Gamma\!\cup\!\{L,R\})$, kde $L,R\notin\Gamma$, je parciální} přechodová funkce, \emph{a}
\item \emph{$q_0 \in Q$ je} počáteční stav \emph{a $q_f \in Q$ je} koncový stav.
\end{itemize}

Symbol $\Delta$ značí tzv. \emph{blank} (prázdný symbol), který se vyskytuje na místech pásky, která ještě nebyla použita.

\emph{Konfigurace pásky} se skládá z~nekonečného řetězce, který reprezentuje obsah pásky a~pozice hlavy na tomto řetězci. Jedná se o~prvek množiny $\{\gamma \Delta^\omega | \gamma \in \Gamma^*\} \times \mathbb{N}$\footnote{Pro libovolnou abecedu $\Sigma$ je $\Sigma^\omega$ množina všech \emph{nekonečných} řetězců nad $\Sigma$, tj. nekonečných posloupností symbolů ze $\Sigma$.}.
\emph{Konfiguraci pásky} obvykle zapisujeme jako $\Delta x y z \underline{z} x \Delta \ldots$ (podtržení značí pozici hlavy). \emph{Konfigurace stroje} je pak dána stavem řízení a konfigurací pásky. Formálně se jedná o prvek množiny $Q \times \{\gamma \Delta^\omega | \gamma \in \Gamma^*\} \times \mathbb{N}$.
\end{definice}

\subsection{Podsekce obsahující větu a odkaz}
\begin{definice}
\label{definice2} Řetězec $w$ nad abecedou $\Sigma$ je přijat TS \emph{$M$ jestliže $M$ při aktivaci z počáteční konfigurace pásky $\underline{\Delta} w \Delta \ldots$ a počátečního stavu $q_0$ zastaví přechodem do koncového stavu $q_F$, tj. ($q_0, \Delta w \Delta^\omega, 0) \stackrel[M]{*}{\vdash} (q_F, \gamma, n)$ pro nějaké $\gamma \in \Gamma^* $ a $ n \in \mathbb{N}$.}

\emph{Množinu $L(M) = \{w\:|\:w$ je přijat TS $M$\} $\subseteq \Sigma^*$ nazýváme} jazyk přijímaný TS $M$.
\end{definice}

Nyní si vyzkoušíme sazbu vět a~důkazů opět s~použitím balíku \texttt{amsthm}.

\begin{veta}
\label{Věta 1.} \emph{Třída jazyků, které jsou přijímány TS, odpovídá} rekurzivně vyčíslitelným jazykům.
\end{veta}

\begin{proof}
V~důkaze vyjdeme z Definice \ref{definice1} a \ref{definice2}.
\end{proof}

\section{Rovnice}
Složitější matematické formulace sázíme mimo plynulý text. Lze umístit několik výrazů na~jeden řádek, ale pak je třeba tyto vhodně oddělit, například příkazem \verb|\quad|.


$$\sqrt[i]{x^3_i} \quad \text{kde } x_i \text{ je } i\text{-té sudé číslo}\quad y^{2 \cdot y_i}_i \neq y^{y^{y_i}_i}_i$$


V rovnici (\ref{rovnice1}) jsou využity tři typy závorek s různou explicitně definovanou velikostí.

\begin{equation}
x = \bigg\{\Big(\big[a + b \big] * c \Big)^d \oplus 1 \bigg\}\label{rovnice1}
\end{equation}
\begin{equation}
y = \lim_{x\to\infty} \frac{\sin^2 x + \cos^2 x}{\frac{1}{\log_{10}x}}\label{rovnice2}
\end{equation}

V~této větě vidíme, jak vypadá implicitní vysázení limity $\lim_{n\to\infty}f(n)$ v~normálním odstavci textu. Podobně je to i~s~dalšími symboly jako $\sum_{i=1}^{n} 2^i$ či $\bigcap_{A \in \mathcal{B}} A$. V~případě vzorců $\lim\limits_{n\to\infty}f(n)$ a $\sum\limits_{i=1}^{n} 2^i$ jsme si vynutili méně úspornou sazbu příkazem \verb|\limits|.

\vfill
\section{Matice}
Pro sázení matic se velmi často používá prostředí \texttt{array} a závorky (\verb|\left|, \verb|\right|).
$$
\left( \begin{array}{ccc}
a + b & \widehat{\xi + \omega} & \hat{\pi}\\
\vec{\mathbf{a}} & \overleftrightarrow{AC} & \beta
\end{array}\right)
= 1 \iff \mathbb{Q} = \mathcal{R}
$$

Prostředí \texttt{array} lze úspěšně využít i jinde.

$$
\binom{n}{k} =
\left\{\begin{array}{c l}
0 & \text{pro } k < 0 \text{ nebo } k > n \\
\frac{n!}{k!(n-k)!} & \text{pro } 0 \leq k \leq n.
\end{array} \right. $$

\end{document}